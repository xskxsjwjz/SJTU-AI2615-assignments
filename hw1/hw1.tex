%Example of use of oxmathproblems latex class for problem sheets
\documentclass{oxmathproblems}

%(un)comment this line to enable/disable output of any solutions in the file
%\printanswers

%define the page header/title info
\course{Algorithm Design and Analysis}
\sheettitle{Assignment 1 \\ Deadline: Oct 17, 2024} %can leave out if no title per sheet


\begin{document}

\begin{questions}
\miquestion[25]
Asymptotic notations.
\begin{parts}
    \part In each of the following situations, indicate whether $f=O(g)$, or $f=\Omega(g)$, or both (in which case $f=\Theta(g)$). Justify your answer.
    \begin{enumerate}
        \item $f(n)=n^{1/2}$ and $g(n)=5^{\log_2n}$
        \item $f(n)=100n+\log n$ and $g(n)=n+(\log n)^2$
        \item $f(n)=(\log n)^{\log n}$ and $g(n)=n/\log n$
        \item $f(n)=(\log n)^{\log n}$ and $g(n)=2^{(\log_2n)^2}$
        \item $f(n)=\sum_{i=1}^ni^k$ and $g(n)=n^{k+1}$
    \end{enumerate}
    \part Let $f(n)=\left(2\cdot \lceil \frac n2\rceil\right)!$ and $g(n)=\left(2\cdot \lfloor \frac n2\rfloor+1\right)!$. Prove that neither $f=O(g)$ nor $f=\Omega(g)$ is true.
\end{parts}

\miquestion[25]
Prove the following generalization of the master theorem. Given constants $a\geq 1,b> 1,d\geq 0$, and $w\geq 0$, if $T(n)=1$ for $n<b$ and $T(n) = aT(n/b) + n^d\log^w n$, we have
  $$
    T(n) = \begin{cases}
      O(n^d\log^w n) & \mbox{if }a < b^d \\
      O(n^{\log_b a}) & \mbox{if }a > b^d \\
      O(n^d\log^{w+1} n) & \mbox{if }a = b^d
    \end{cases}.
  $$

\miquestion[25]
For two vectors $\mathbf{a}=(a_1,\ldots,a_d),\mathbf{b}=(b_1,\ldots,b_d)\in\mathbf{R}^d$, we say $\mathbf{a}$ is \emph{greater than} $\mathbf{b}$ if $a_k>b_k$ for each $k=1,\ldots,d$.
You are given two collections of vectors $A,B\subseteq\mathbb{R}^d$.
The objective is to count the number of pairs $(\mathbf{a},\mathbf{b})\in(A,B)$ such that $\mathbf{a}$ is greater than $\mathbf{b}$.
You can assume all the entries in all the vectors are distinct.
Let $n=|A|+|B|$ and $d$ be the dimension of the vectors.

\begin{parts}
\part Design an $O(n\log n)$ time algorithm for this problem with $d=1$.
\part Design an algorithm for this problem with $d=2$. Your algorithm must run in $o(n^{1.1})$ time.
\part Generalize the algorithm in part (b) so that it works for general $d$. Analyze its running time. The running time must be in terms of $n$ and $d$.
\end{parts}

\miquestion[25]
Let $A$ be a square matrix. This question discusses the computation of $A^2$.
\begin{parts}
    \part Show that five multiplications are sufficient to compute the square of a $2\times 2$ matrix.
    \part What is wrong with the following algorithm for computing the square of an $n\times n$ matrix?
    \begin{itemize}
        \item Use a divide-and-conquer approach as in Strassen's algorithm, except that instead of getting $7$ subproblems of size $n/2$, we now get $5$ subproblems of size $n/2$ thanks to part (a). Using the same analysis as in Strassen's algorithm, we can conclude that the algorithm runs in time $O(n^{\log_25})$.
    \end{itemize}
    \part In fact, squaring matrices is no easier than matrix multiplication. In this part, you will show that if $n\times n$ matrices can be squared in time $S(n)=O(n^c)$, then any two $n\times n$ matrices can be multiplied in time $O(n^c)$.
    \begin{enumerate}
        \item Given two $n\times n$ matrices $A$ and $B$, show that the matrix $AB+BA$ can be computed in time $3S(n)+O(n^2)$.
        \item Given two $n\times n$ matrices $X$ and $Y$, define the $2n\times 2n$ matrices $A$ and $B$ as follows:
        $$A=\left[\begin{array}{cc}
            X & 0 \\
            0 & 0
        \end{array}\right]\qquad\mbox{and}\qquad B=\left[\begin{array}{cc}
            0 & Y \\
            0 & 0
        \end{array}\right].$$
        What is $AB+BA$, in terms of $X$ and $Y$?
        \item Using 1 and 2, argue that the product $XY$ can be computed in time $3S(2n)+O(n^2)$. Conclude that matrix multiplication takes time $O(n^c)$.
    \end{enumerate}
\end{parts}
  
\miquestion
How long does it take you to finish the assignment (including thinking and discussion)?
Give a score (1,2,3,4,5) to the difficulty.
Do you have any collaborators?
Please write down their names here.
 

\end{questions}


\end{document}
