\documentclass{oxmathproblems}

\course{Algorithm Design and Analysis}
\sheettitle{Assignment 1 \\Wu Jiabao\\ Deadline: Oct 17, 2024}

\begin{document}

\begin{questions}
\miquestion
% Asymptotic notations.
\begin{parts}
    \part
    \begin{enumerate}
        \item $f=O(g)$. Since $g(n)=n^{\log _25}$, $\lim\limits_{n\to\infty}\frac{f(n)}{g(n)}
        =0<\infty$

        \item $f = \Theta (g)$. Since $\lim\limits_{n\to \infty}\frac{f(n)}{g(n)}
        =100\in(0,\infty)$.

        \item $f=\Omega (g)$. Since $f(n)=e^{\log n\log\log n},g(n)=e^{\log n-\log\log n}$, 
        $\lim\limits_{n\to \infty}\frac{f(n)}{g(n)}=\infty>0$.

        \item $f=O(g)$. Since $f(n)=e^{\log n\log\log n},g(n)=e^{\log n\log_2n}$, 
        $\lim\limits_{n\to \infty}\frac{f(n)}{g(n)}=0<\infty$.

        \item $f=O(g)$. Since $f(n)=\Sigma^n_{i=1}i^k\le n*n^k=g(n)$ for any $n>n_0=1$, 
        $\lim\limits_{n\to \infty}\frac{f(n)}{g(n)}\le 1<\infty$.
    \end{enumerate}
    \part
    Obviously:
    $$
      \frac{f(n)}{g(n)} = \begin{cases}
          n+1, &\mbox{n is odd},\\
          \frac{1}{n+1}, &\mbox{n is even}
      \end{cases}.
    $$
    Suppose $f=O(g)$ is true. Then there exists $n_0\in \mathbb{Z^+},c\in \mathbb{R^+}$ such that $\frac{f(n)}{g(n)}\le c$ for any $n>n_0$. 
    But $\forall n=2k+1$, where $k\in \mathbb{N^+}$, $\lim\limits_{n\to \infty}\frac{f(n)}{g(n)}=\infty$, which causes a contradiction.

\end{parts}
\miquestion
The running time of solving all size-1 problem is:
$$
a^{\log_bn}\cdot O(1)=O(n^{\log_ba})
$$
The total combining is:
$$
c\cdot (n^d\log^wn)+ca\cdot\left((\frac{n}{b})^d\log^w\frac{n}{b}\right)
+\dots+ca^k\left((\frac{n}{b^k})^d\log^w\frac{n}{b^k}\right)+\dots+a^{\log_bn}\cdot c
$$
Since $\log^w\frac{n}{b^k}\le\log^wn$, The simplification is:
$$
T(n)\le cn^d\log^wn(1+\frac{a}{b^d}+\dots+(\frac{a}{b^d})^{\log_bn})
$$

Thus $T(n)=O(n^d\log^wn)(1+\frac{a}{b^d}+\dots+(\frac{a}{b^d})^{\log_bn})$.

if $\frac{a}{b^d}<1$, $T(n)=O(n^d\log^wn)$;

if $\frac{a}{b^d}=1$, $T(n)=O(n^d\log^wn)(\log_bn+1)=O(n^d\log^{w+1}n)$;

if $\frac{a}{b^d}>1$, the final term conquers the sum. $T(n)\le c\cdot a^{\log_bn}=O(n^{\log_ba})$.

\miquestion
\begin{parts}
    \part
    \begin{enumerate}
        \item Construct set $C=A\cup B$, thus $|C|=n$. Then sort $C$ in descending order by merge sort, which causes $O(n\log n)$ time.
        \item Iterate through sorted $C$ from head to end and let $m=|B|$, $res=0$ at first, which causes $O(n)$ time.
        \item For each vector $i$ in sorted $C$, if $i\in A$, $res=m+res$; If $i\in B$, $m=m-1$.
        \item print $res$.
    \end{enumerate}
    It's time complexity is $T(n)=O(n\log n)+O(n)=O(n\log n)$.

    \part
    count($A$, $B$):
    Use median-of-medians to find the median of x's value from $A\cup B$, denoted $x_0$;\\
    Then divide $A$ and $B$ into four parts:\\
    $A_1=\{(x,y)|x\le x_0,(x,y)\in A\}\\
    A_2=\{(x,y)|x> x_0,(x,y)\in A\}\\
    B_1=\{(x,y)|x\le x_0,(x,y)\in B\}\\
    B_2=\{(x,y)|x> x_0,(x,y)\in B\}$\\
    Then we compare $A_2$ and $B_1$'s y's value. Use descending order sort $A_2\cup B_1$ and iterate their y's value.
    As it is claimed in $(a)$, it takes $O(n\log n)$ time.\\
    Finally we recursively divide $A_2B_2$ and $A_1B_1$ as we did above. Thus it turns $T(n)$ into $2T(\frac{n}{2})+O(n\log n)$.\\
    According to $2$ we can claim that the total time complexity is: 
    $$T(n)=O(n\log^2n)=O(n^{1.1})$$

    \part
    We discuss about $d=3$ situation.\\
    As the same above, we find the median of the first dimention and divide $A\cup B$ into two parts;
    Then for $A_2$ and $B_1$, we only need to consider them in 2 dimentions.\\
    Thus it time complexity is: $$T(n)=2T(\frac{n}{2})+O(n\log^{d-1}n)=O(n\log^dn)$$
\end{parts}
\miquestion
\begin{parts}
    \part
    Since $\begin{bmatrix}a&b\\c&d\end{bmatrix}^2=\begin{bmatrix}a^2+bc&b(a+d)\\c(a+d)&bc+d^2\end{bmatrix}$,
    the computation of$A^2$ requires to compute $a^2,d^2,bc,b(a+d),c(a+d)$.
    \part
    In$(a)$ we find that $bc,b(a+d),c(a+d)$ is not the square of a $n\times n$ matrix, thus it can not be computed in 5 multiplications.
    \part
    \begin{enumerate}
        \item $AB+BA=(A+B)^2-A^2-B^2$, thus it takes $3S(n)+O(n^2)$ to compute.
        \item $AB+BA=\begin{bmatrix}0&XY\\0&0\end{bmatrix}$.
        \item Since reading the numbers takes $O(n^2)$ time, $c>2$. Then $3S(2n)+O(n^2)=3O((2n)^c)+O(n^2)=O(n^c)$.
    \end{enumerate}
\end{parts}
\miquestion
It takes me about 8 hours to finish the assignment. The difficulty is 4 according to me.\\
I discussed with my roommates, they are Wang Kun and Li Haochen.
\end{questions}
\end{document}
