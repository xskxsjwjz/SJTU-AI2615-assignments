\documentclass{oxmathproblems}

\course{Algorithm Design and Analysis}
\sheettitle{Assignment 2 \\ Wu Jiabao 523030910241}

\begin{document}
\begin{questions}

\miquestion
Firstly we construct $G$'s reverse graph $G'$, and employ DFS on it with finish time.
DFS $G$ by descending order of the finish time. Then we've got SCCs of $G$.
Let each SCC in $G$ be a super vertex, and these vertices form a new graph $G^*$.
DFS $G^*$ and if it finds a super vertex $u$ such that it is not reachable to the super vertex containing $t$, 
the required path from $s$ to $t$ doesn't exist. Otherwise the path can be found.
\par
Prove. The SCCs of $G$ form a partition. for each SCC, we can explore all the vertices in it if we can reach one of these points. 
Thus we can seem the SCC as a super vertex. During DFS, if we find a super vertex $u$ such that it has no not-explored-yet neighbors and $t\notin u$, 
we can conclude that after exploring $u$, we cannot explore the SCC containing $t$.
\par
Time complexity: Employing DFS for three times, constructing new graph and checking reachability. The final complexity is:
$$O(|V|+|E|)+O(|E|)+O(|V|+|E|)=O(|V|+|E|)$$

\miquestion
\begin{enumerate}
    \item If G is a tree:\\
Suppose there are two paths from $s$ to a point $u$ in $G$: $(s,x_1),(x_1,x_2)\dots,(x_n,u)$ and $(s,y_1),(y_1,y_2)\dots,(y_n,u)$.
Then obviouly there exists a path: $(s,x_1),(x_1,x_2)\dots,\\
(x_n,u),(u,y_n),\dots,(y_1,s)$, which is a cycle. Thus $\forall v\in V$, there is only a unique path from $s$ to $v$.\\
For either DFS or BFS, it explores $v$ along the path from $s$ to $v$. Therefore, the parent-child relationship established during
DFS and BFS are the same, leading to identical results.
    \item If DFS and BFS trees are identical:\\
Assume $G$ is not a tree, then there is a cycle in $G$, then there exists at least one point in the cycle, there are at least two paths from $s$ to it.
For DFS, it will trace deep into the path; For BFS, it will spread its exploring range.
The difference in exploring and multiple paths to a same point lead to different results. Contradiction!
\end{enumerate}
Thus the DFS and BFS trees rooted at $s$ are identical if and only if $G$ is a tree.

\miquestion
We use Dijkstra Algorithm.
\begin{enumerate}
    \item Let $T=\{s\}$, $tdist[s]\leftarrow0,\ tdist[v]\leftarrow \infty$ for all $v$ other than $s$. $tdist[v]\leftarrow w(s,v)$ for all $(s,v)\in E$.
    \item $B[u]\leftarrow false$ for all $u\in V$.
    \item Find $v\notin T$ with smallest $tdist[v]$, $T\leftarrow T\cup \{v\}$.
    \item $tdis[u] \leftarrow min\{tdist[u],\ tdist[v]+w(v,u)\}$ for all $(v,u)\in E$.
    \item If $tdist[u]=tdist[v]+w(u,v)$, $B[u]=true$.
\end{enumerate}
Prove of correctness.
\begin{enumerate}
    \item If we employ the algorithm above but find $B[u]=true$ and the shortest path from $s$ to $u$ is unique. 
Then there exists $v$ such that $tdist[u]=tdist[v]+w(u,v)$. Thus we can go from $s$ to $v$ than $u$, or go from $s$ to $u$ by another path. Otherwise when judging whether $tdist[u]=tdist[v]+w(u,v)$, we will find that $tdist[u]=\infty\neq tdist[v]+w(u,v)$. Contradiction!
    \item If we find $B[u]=false$ but the path is not unique, then $\forall(u,v)\in E$, $tdist[u]\neq tdist[v]+w(u,v)$.
Thus the shortest path is always unique.
\end{enumerate}
The time complexity of Dijkstra Algorithm: $O((|E|+|V|)\log |V|)$.

\miquestion
\begin{parts}
\part 
\begin{enumerate}
    \item We use DFS to find the topological order of $G$.
    \item Then let $dist[s] \leftarrow 0, dist[v] \leftarrow \infty$ for all $v$ other than $s$.
    \item Iterate $u \in V$ by the topological order: \\
    For all $(u,v) \in E, dist[v] = min(dist[v],dist[u] + w(u,v))$.
    \item output dist.
\end{enumerate}
Prove of correctness. Since we iterate $V$ by topological order, there will not exist
a shorter path to $u$ after iterating $u$. Since we are using Bellman-Ford Algorithm
to culculate the distance to each point, the culculation result is correct.
The time complexity is: Finding topological order $O(|V | + |E|)$; Iterating edges
and vertices: $O(|V | + |E|)$. So it is:
$O(|V | +|E|)$.
\part Yes, we can. Prove.
Suppose we have a path $P = \Sigma w_i$ from s to u in G, then we negate every weight
to form G′, we have P′ = −Σiwi. Thus if we find the longest path in G′, we find
the shortest path in G. We can use the algorithm in (a) to solve it
\end{parts}
\end{questions}

\end{document}