%Example of use of oxmathproblems latex class for problem sheets
\documentclass{oxmathproblems}
\usepackage{algorithm}
\usepackage{algpseudocode}

%(un)comment this line to enable/disable output of any solutions in the file
%\printanswers

%define the page header/title info
\course{Algorithm Design and Analysis}
\sheettitle{Assignment 5 \\ Deadline: Jan 2, 2025} %can leave out if no title per sheet


\begin{document}


\begin{questions}

\miquestion[25] \textbf{[Invertible Matrices]}
Given an $n\times n$ matrix $A$ whose entries are either $0$ or $1$, you are allowed to choose a set of entries with value $1$ and modify their value to $0$. Design a polynomial-time algorithm to decide if we can make $A$ invertible by the above-mentioned modification. Prove the correctness of your algorithm and analyze its time complexity. (Hint: Prove that this is possible if and only if there exist $n$ entries with value $1$ such that no two entries are in the same row and no two entires are in the same column.)

\miquestion[25] \textbf{[Common System of Distinct Representatives]}
Given a ground set $U=\{1,\ldots,n\}$ and a collection of $k$ subsets $\mathcal{A}=\{A_1,\ldots,A_k\}$, a \emph{system of distinct representatives} of $\mathcal{A}$ is a ``representative'' collection $T$ of distinct elements from the sets in $\mathcal{A}$.
Specifically, we have $|T|=k$, and the $k$ \emph{distinct} elements in $T$ can be ordered as $u_1,\ldots,u_k$ such that $u_i\in A_i$ for each $i=1,\ldots,k$.
For example, $\{A_1=\{2,8\},A_2=\{8\},A_3=\{4,5\},A_4=\{2,4,8\}\}$ has a system of distinct representatives $\{2,4,5,8\}$ where $2\in A_1,4\in A_4,5\in A_3,8\in A_2$, while $\{A_1=\{2,8\},A_2=\{8\},A_3=\{4,8\},A_4=\{2,4,8\}\}$ does not have a system of distinct representatives.
\begin{parts}
\part[10] Design a polynomial time algorithm to decide if $\mathcal{A}$ has a system of distinct representatives.
\part[15] Given a ground set $U=\{1,\ldots,n\}$ and two collections of $k$ subsets $\mathcal{A}=\{A_1,\ldots,A_k\}$ and $\mathcal{B}=\{B_1,\ldots,B_k\}$, a \emph{common system of distinct representatives} is a collection $T$ of $k$ elements that is a system of distinct representatives of both $\mathcal{A}$ and $\mathcal{B}$. Design a polynomial time algorithm to decide if $\mathcal{A}$ and $\mathcal{B}$ have a common system of distinct representatives.
\end{parts}
For each part, prove the correctness of your algorithm, and analyze its time complexity.

\miquestion[25] \textbf{[Perfect Matching on Bipartite Graph]}
A graph is \emph{regular} if every vertex has the same degree.
Let $G=(A,B,E)$ be a regular bipartite graph with $|E|>0$.
\begin{parts}
\part[10] Prove that $|A|=|B|$.
\part[15] Let $n=|A|=|B|$. Prove that $G$ must contain a matching of size $n$.
\end{parts}

\miquestion[25] \textbf{[LP-Duality and Total Unimodularity]}
In this question, we will prove K\"{o}nig-Egerv\'{a}ry Theorem, which states that, in any bipartite graph, the size of the maximum matching equals to the size of the minimum vertex cover.
Let $G=(V,E)$ be a bipartite graph.
\begin{parts}
\part (4 points) Explain that the following linear program describes \emph{the fractional version of} the maximum matching problem (i.e., replacing $x_e\geq 0$ to $x_e\in\{0,1\}$ gives you the exact description of the maximum matching problem).
\begin{align*}
\text{maximize }& \sum_{e\in E}x_e \\
\text{subject to }&\sum_{e:e=(u,v)}x_e\leq 1  \tag{$\forall v\in V$}\\
&x_e\geq0\tag{$\forall e\in E$}
\end{align*}
\part[4] Write down the dual of the above linear program, and justify that the dual program describes the fractional version of the minimum vertex cover problem.
\part[7] Show by induction that the \emph{incident matrix} of a bipartite graph is totally unimodular. (Given an undirected graph $G=(V,E)$, the incident matrix $A$ is a $|V|\times|E|$ zero-one matrix where $a_{ij}=1$ if and only if the $i$-th vertex and the $j$-th edge are incident.)
\part[6] Use results in (a), (b) and (c) to prove K\"{o}nig-Egerv\'{a}ry Theorem.
\part[4] Give a counterexample to show that the claim in K\"{o}nig-Egerv\'{a}ry Theorem fails if the graph is not bipartite.
\end{parts}

\miquestion[Bonus 60]
\textbf{[Dinic's Algorithm]}
In this question, we are going to work out \emph{Dinic's algorithm} for computing a maximum flow.
Similar to Edmonds-Karp algorithm, in each iteration of Dinic's algorithm, we update the flow $f$ by increasing its value and then update the residual network $G^f$.
However, in Dinic's algorithm, we push flow along \emph{multiple} $s$-$t$ paths in the residual network instead of a \emph{single} $s$-$t$ path as it is in Edmonds-Karp algorithm.

In each iteration of the algorithm, we find the \emph{level graph} of the residual network $G^f$.
Given a graph $G$ with a source vertex $s$, its level graph $\overline{G}$ is defined by removing edges from $G$ such that only edges pointing from level $i$ to level $i+1$ are kept, where vertices at level $i$ are those vertices at distance $i$ from the source $s$.
An example of level graph is shown in Fig.~\ref{fig:levelgraph}.

Next, the algorithm finds a \emph{blocking flow} on the level graph $\overline{G}^f$ of the residual network $G^f$.
A blocking flow $f$ in $G$ is a flow such that each $s$-$t$ path contains at least one \emph{critical edge}.
Recall that an edge $e$ is \emph{critical} if the amount of flow on it reaches its capacity: $f(e)=c(e)$.
Fig.~\ref{fig:blockingflow} gives examples for blocking flow.

Dinic's algorithm is then described as follows.
\begin{enumerate}
    \item Initialize $f$ to be the empty flow, and $G^f=G$.
    \item Do the following until there is no $s$-$t$ path in $G^f$:
    \begin{itemize}
        \item construct the level graph $\overline{G}^f$ of $G^f$.
        \item find a blocking flow on $\overline{G}^f$.
        \item Update $f$ by adding the blocking flow to it, and update $G^f$.
    \end{itemize}
\end{enumerate}

\begin{figure}[ht]
    \centering
    \includegraphics[width=\textwidth]{levelgraph.pdf}
    \caption{The graph shown on the right-hand side is the level graph of the graph on the left-hand side. Only edges pointing to the next levels are kept. For example, the edges $(c,a)$ and $(b,e)$ are removed, as they point at vertices at the same level. If there were edges pointing at previous levels, they should also be removed.}
    \label{fig:levelgraph}
\end{figure}

\begin{figure}[ht]
    \centering
    \includegraphics[width=\textwidth]{blockingflow.pdf}
    \caption{Blocking flow examples}
    \label{fig:blockingflow}
\end{figure}

Complete the analysis of Dinic's algorithm by solving the following questions.
\begin{parts}
\part[15] Prove that, after each iteration of Dinic's algorithm, the distance from $s$ to $t$ in $G^f$ is increased by at least $1$.
\part[15] Design an $O(|V|\cdot |E|)$ time algorithm to compute a blocking flow on a level graph.
\part[10] Show that the overall time complexity for Dinic's algorithm is $O(|V|^2\cdot|E|)$.
\part[20] \textbf{(challenging)} We have seen in the class that the problem of finding a maximum matching on a bipartite graph can be converted to the maximum flow problem. Show that Dinic's algorithm applied to finding a maximum matching on a bipartite graph only requires time complexity $O(|E|\cdot\sqrt{|V|})$.
\end{parts}
  
\miquestion
How long does it take you to finish the assignment (including thinking and discussion)?
Give a score (1,2,3,4,5) to the difficulty.
Do you have any collaborators?
Please write down their names here.
 

\end{questions}


\end{document}
