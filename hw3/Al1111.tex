\documentclass{oxmathproblems}
\usepackage{algorithm}
\usepackage{algpseudocode}

\course{Algorithm Design and Analysis}
\sheettitle{Assignment 3 \\ Wu Jiabao}

\begin{document}

\begin{questions}
\miquestion
\begin{parts}
    \part Suppose we have two maximal independent sets $A,\ B$ with $|A|<|B|$.\\
    There exists $x\in B\setminus A$, $\{x\}\cup A\in \mathcal{I}$. Since $A\subsetneq \{x\}\cup A$, this is a contradiction.
    Thus $|A|=|B|$.

    \part $\textbf{(hereditary property)}$. Obviously $S$ is nonempty. Since $\forall F\in S$, take $F'\subseteq F$, we have $F'$ 
    does not contain a cycle, thus $F'\in S$.\\
    $\textbf{(exchange property)}$. Suppose $|A|<|B|$. If $x_0\in B\setminus A$ is connected to a point that $A$ doesn't reach, 
    take $\{x_0\}\cup A$ and it must not have a cycle.\\
    If we cannot find such an $x_0$ above, we have $A$ is connected to the same points as $B$. 
    Since $|A|<|B|$, $B$ contains at least one more edge than $A$. This condition only happens when $A$ is not a single tree. 
    Since $A$ contains at least two trees, the project edge $x_0$ must connect two trees in $A$. (otherwise there must have a cycle in $B$)\\
    Hence $\{x_0\}\cup A\in \mathcal{I}$, $M$ is a matroid.\\
    All the spanning trees without vertices are the maximal sets.

    \part we know $\{x\}\in \mathcal{I}$.
    Assume a maximal set $S_0$ with maximal weight not containing $x$ and $S'=\{x\}$. \\
    If $|S_0|=1$, obviously $w(x)$ is the optimal. Contradiction!\\
    If $|S_0|>1$, then from exchange property we know there exists $x_0\in S_0$ that $S'=S'\cup\{x_0\}\in\mathcal{I}$. 
    If $|S'|<|S_0|$, repeate the process till $S'$ is maximal. Now since $|S_0|=|S'|$, let $\{y\}=S_0\setminus S'$. 
    Obviously $w(S')-w(S_0)=w(x)-w(y)>0$, hence $S_0$ is not the optimal set. Contradiction!\\
    If there exist $w(y)=w(x)$, $S_0$ and $S'$ are both optimal sets.\\
    Therefore, there must exist a maximal set with maximal weight containing $x$.

    \part We proof it by induction. Let $S$ be the subset of an optimal set.\\
    Assume after adding the (i-1)-th element, $S$ is the subset of an optimal set. 
    The first step is proved in (c).\\
    Now we proof $S\cup \{x_i\}$ is the subset of an optimal set. Suppose there is an optimal set that doesn't contain $x_i$. 
    Then we can construct a maximal set $S'=S^*\setminus \{y\}\cup\{x_i\}$. If $w(y)<w(x)$, $S'$ is the optimal set. Contradiction! 
    If $w(y)=w(x)$, $S'$ is also an optimal set. Since $S\subset S^*$, we have $S\subset S'$. 
    Since $\{x_i\}\in S'$, we have $S\cup\{x_i\}\subset S'$.\\
    Thus the algorithm always return the optiaml set.
    \newpage
    \part Construct set $\mathcal{I}=\{F\subset U\mid\mbox{all vectors in F are linearly independent.}\}$.\\
    Thus for any subset of $F$, the vectors it contains are linearly independent.\\
    Take $A,B\in\mathcal{I}$ with $|A|<|B|$. Suppose $\forall b\in B$, $b$ is linearly correlated with a corresponding vector $a\in A$. 
    Since $b_1,b_2\in B$ cannot be linearly correlated with a same vector in $A$(otherwise $b_1,b_2$ are linearly correlated), 
    there must be $b_{|B|}$ with no corresponding linearly correlated vector, thus $b_{|B|}\cup A\in\mathcal{I}$.\\
    Thus $M=(U,\mathcal{I})$ is a matroid. Then use the algorithm given in (c) to find the optimal set.
\end{parts}

\miquestion
Select an arbitrary vertex as root, then determine the level of each vertex by DFS. The root's level is $1$, its children's level is $2$ and so on.
\begin{enumerate}
    \item $S\gets\emptyset$.
    \item Each time select an uncovered leaf with maximal level:
    \item \hspace{0.5cm} Find its k-th ancestor $x$. If k-th ancestor doesn't exist, choose root as $x$.(if $k=2$ then just find its grandparent.)
    \item \hspace{0.5cm} $S\gets S\cup\{x\}$, add all the vertices to $U$ that are within distance $k$ from $x$.
    \item Until $U=V$.
\end{enumerate}
\textbf{Proof of Correctness: }\\
\textbf{Basic Step: }Assume the first element we add to $S$ is $x_0$. If $\{x_0\}$ is not a subset of any optimal set, then they must contain $x_0$'s descendants.(otherwise the leaves won't be contained.)\\
If $x_0$ has multiple children, choosing its descendants causes $S$ larger because if we choose a descendants in one branch, the leaves in other branches cannot be covered. Thus $x_0$ is a better choice.\\
If $x_0$ has a single child, choosing its child cannot contain $x_0$'s k-th ancestor, thus $x_0$ is a better choice.\\
If $x_0$ is root, we have all the vertices are within distance of $k$ from $x$.\\
Thus $\{x_0\}$ is the subset of some optimal set.\\
\textbf{Induction Hypothesis: }Suppose after i-th iteration we have $S$ and $S\subset S^*$, $S^*$ is an optimal set.\\
The (i+1)-th selection adds $x$ to $S$. Suppose there is no optimal set that contains $S\cup\{x\}$.\\
Similar to basic step, these optimal sets contain $x$'s descendants, which causes contradiction. Thus $S\cup\{x\}$ is a subset of some optimal sets.
\par
\textbf{Time Complexity: }
DFS: $O(|V|+|E|)$\\
Sort vertices into decreasing order of level: $O(|V|\log |V|)$\\
Add vertices into $U$: $O(|V|)$\\
Total time complexity: $O(|V|\log |V|)$

\miquestion
\begin{parts}
    \part \textbf{Initialize: } Employ DFS on $G$ for first. During DFS we count the amount of CCs and vertices in each CC(denoted as $w(CC)$). The set $C$ includes all CCs that DFS finds. Let $i=1,\ S=\emptyset$.\\
    Sort $C$ in decreasing order by weight $w$.\\
    \textbf{For} $x\in C$ in decreasing order of $w$:\\
    \textbf{If} $i<k$: Take an arbitrary vertex $a$ in $x$ and $S\gets S\cup\{a\}$, $i\gets i+1$.\\
    \textbf{If} $i=k$: \textbf{endfor}.\\
    \textbf{endfor}\\
    \textbf{return} $S$\\
    The algorithm takes $O(|V|+|E|)$ time.

    \part Denote $f(S)=w(S)=\sum_{x\in S}w(x)$. Firstly we employ DFS to find SCCs in $G$ and count the vertices each SCC contains as $w(SCC)$, then denote each SCC as a super node. These SCCs construct a new graph $G^*$, which is a DAG. For each super node in $G^*$, if its in-degree is 0, it is a sourse. 
    For each sourse we employ DFS on it to find the super nodes it can reach. Let the set $A$ contains each sourse and super nodes it can reach. Let $w(A)=\sum_{x\in A}w(x)$ and $T$ contains all these $A$. $U$ contains all these super nodes in $G^*$.
    \begin{enumerate}
        \item Initialize $S\gets\emptyset$
        \item Repeate the followings:
        \item \hspace{0.5cm} find $A\in T\setminus S$ that maximizes $f(S\cup\{A\})-f(S)$
        \item \hspace{0.5cm} update $S\gets S\cup\{A\}$.
        \item Until $f(S)=|V|$ or $|S|=k$.
    \end{enumerate}
    \textbf{Proof of Correctness: }\\
    Suppose the optimal set is $S_{OPT}$, each element of it covers $\frac{1}{k}$ fraction.\\
    Let $S=\{A_1\dots A_k\}$ be the output of the algorithm.\\
    $\{A_1\}$ covers $\frac{1}{k}$ fraction, $\{A_1,A_2\}$ covers $1-(1-\frac{1}{k})^2$ fraction.\\
    Thus $f(S)\ge\left(1-(1-\frac{1}{k})^k\right)f(S_{OPT})\ge\left(1-\frac{1}{e}\right)f(S_{OPT})$.\\
    Thus this is a ($1-\frac{1}{e}$)-approximation.

    \part Let $U=V$ be the ground set of vertices. Consider there donot exist SCC in $G$ so $G$ is a DAG. We can find sourses in $G$ denoted as $s$ by topological order. 
    Then the max-k-coverage problem can be view as to minimize the sourses we choose and maximize the vertices these chosen DAGs contain, which is a special case of the maximum reachability problem.\\
    Thus the maximum reachability problem is NP-hard.
\end{parts}
\miquestion
It takes me 14 hours to finish the work. Difficulty is 5. Collaborators: Li Haochen, Yu Junjie.
\end{questions}
\end{document}